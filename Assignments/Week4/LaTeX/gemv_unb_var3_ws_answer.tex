\documentclass[12pt]{article}

\usepackage{amssymb}
\usepackage{ifthen}
\usepackage[table]{xcolor}
\usepackage{minitoc}
\usepackage{array}

\definecolor{yellow}{cmyk}{0,0,1,0}
\renewcommand{\arraystretch}{1.4}
\newcommand{\R}{\mathbb{R}}

\usepackage{colortbl}

% Page size
\setlength{\oddsidemargin}{-0.5in}
\setlength{\evensidemargin}{-0.5in}
\setlength{\textheight}{10.25in}
\setlength{\textwidth}{7.0in}
\setlength{\topmargin}{-1.35in}

\renewcommand{\arraycolsep}{3pt}


\input color_flatex

\begin{document}
\pagestyle{empty}


\resetsteps % reset all definitions

% Insert output from Spark webpage below


\resetsteps      % Reset all the commands to create a blank worksheet  

% Define the operation to be computed

\renewcommand{\operation}{ y := A x + y }

\renewcommand{\routinename}{\operation}

% Step 1a: Precondition 

\renewcommand{\precondition}{
	y = \widehat{y}
}

% Step 1b: Postcondition 

\renewcommand{\postcondition}{ 
	y = A x + \widehat y
}

% Step 2: Invariant 
% Note: Right-hand side of equalities must be updated appropriately

\renewcommand{\invariant}{
	y = A_L x_T + \widehat y
}

% Step 3: Loop-guard 

\renewcommand{\guard}{
	n( A_L ) < n( A )
}

% Step 4: Initialize 

\renewcommand{\partitionings}{
	$
	A \rightarrow
	\left(\begin{array}{c I c}
	A_L & A_R
	\end{array}\right)
	$
	,
	$
	x \rightarrow
	\left(\begin{array}{c}
	x_{T} \\ \whline 
	x_{B}
	\end{array}\right)
	$
}

\renewcommand{\partitionsizes}{
	$ A_L $ has $ 0 $ columns,
	$ x_T $ has $ 0 $ rows
}

% Step 5a: Repartition the operands 

\renewcommand{\repartitionings}{
	$  \left(\begin{array}{c I c}
	A_L & A_R
	\end{array}\right)
	\rightarrow
	\left(\begin{array}{c I c c}
	A_0 & a_1 & A_2
	\end{array}\right)
	$
	,
	$  \left(\begin{array}{c}
	x_T \\ \whline
	x_B 
	\end{array}\right) 
	\rightarrow
	\left(\begin{array}{c}
	x_0 \\ \whline 
	\chi_1 \\  
	x_2
	\end{array}\right)
	$
}

\renewcommand{\repartitionsizes}{
	$ a_1 $ has $ 1 $ column,
	$ \chi_1 $ has $ 1 $ row}

% Step 5b: Move the double lines 

\renewcommand{\moveboundaries}{
	$  
	A \rightarrow
	\left(\begin{array}{c I c}
	A_L & A_R
	\end{array}\right)
	\leftarrow
	\left(\begin{array}{c c I c}
	A_0 & a_1 & A_2
	\end{array}\right)
	$
	,
	$  \left(\begin{array}{c}
	x_T \\ \whline
	x_B 
	\end{array}\right) 
	\leftarrow
	\left(\begin{array}{c}
	x_0 \\  
	\chi_1 \\ \whline 
	x_2
	\end{array}\right) 
	$
}

% Step 6: State before update
% Note: The below needs editing consistent with loop-invariant!!!

\renewcommand{\beforeupdate}{$ y = A_0 x_0 + \widehat y $}


% Step 7: State after update
% Note: The below needs editing consistent with loop-invariant!!!

\renewcommand{\afterupdate}{$ y = A_0 x_0 + \chi_1 a_1 + \widehat y $}


% Step 8: Insert the updates required to change the 
%         state from that given in Step 6 to that given in Step 7
% Note: The below needs editing!!!

\renewcommand{\update}{
	$
	\begin{array}{l}          % do not delete this line 
	y = \chi_1 a_1 + y
	\end{array}               % do not delete this line 
	$
}



\begin{center}
	\FlaWorksheet
\end{center}

\newpage

\begin{figure}[p]
\begin{center}
	\FlaWorksheetZero
\end{center}
\end{figure}

\begin{figure}[p]
\begin{center}
	\FlaWorksheetOne
\end{center}
\end{figure}

\begin{figure}[p]
\begin{center}
	\FlaWorksheetTwo
\end{center}
\end{figure}

\begin{figure}[p]
\begin{center}
	\FlaWorksheetThree
\end{center}
\end{figure}

\begin{figure}[p]
	\begin{center}
	\FlaWorksheetFour
\end{center}
\end{figure}

\begin{figure}[p]
	\begin{center}
	\FlaWorksheetFive
\end{center}
\end{figure}

\begin{figure}[p]
	\begin{center}
	\FlaWorksheetSix
\end{center}
\end{figure}

\begin{figure}[p]
	\begin{center}
	\FlaWorksheetSeven
\end{center}
\end{figure}

\begin{figure}[p]
	\begin{center}
	\FlaWorksheetEight
\end{center}
\end{figure}

\begin{figure}[p]
	\begin{center}
	\FlaWorksheetNine
\end{center}
\end{figure}

\begin{figure}[p]
\begin{center}
	\FlaAlgorithm
\end{center}
\end{figure}

\end{document}