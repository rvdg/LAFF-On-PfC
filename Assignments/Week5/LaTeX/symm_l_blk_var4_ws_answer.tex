\documentclass[12pt]{article}

\usepackage{amssymb}
\usepackage{ifthen}
\usepackage[table]{xcolor}
\usepackage{minitoc}
\usepackage{array}

\definecolor{yellow}{cmyk}{0,0,1,0}
\renewcommand{\arraystretch}{1.4}
\newcommand{\R}{\mathbb{R}}

\usepackage{colortbl}

% Page size
\setlength{\oddsidemargin}{-0.5in}
\setlength{\evensidemargin}{-0.5in}
\setlength{\textheight}{10.25in}
\setlength{\textwidth}{7.0in}
\setlength{\topmargin}{-1.35in}

\renewcommand{\arraycolsep}{3pt}


\input color_flatex

\begin{document}
\pagestyle{empty}


\resetsteps % reset all definitions

% Insert output from Spark webpage below


\resetsteps      % Reset all the commands to create a blank worksheet  

% Define the operation to be computed

\renewcommand{\operation}{ C := A B + C \mbox{~~~where $ A $ is symmetric and stored in the lower triangular part} }

\renewcommand{\routinename}{\operation}

% Step 1a: Precondition 

\renewcommand{\precondition}{
	C = \widehat{C}
}

% Step 1b: Postcondition 

\renewcommand{\postcondition}{ 
	C = A B + \widehat{C} 
}

% Step 2: Invariant 
% Note: Right-hand side of equalities must be updated appropriately

\renewcommand{\invariant}{
	\left(\begin{array}{c}
		C_T \\ \whline
		C_B 
	\end{array}\right)  = 
	\left(\begin{array}{c}
		A_{TL} B_T + A_{BL}^T B_B + \widehat C_T \\ \whline 
		A_{BL} B_T + {\color{white} A_{BR} B_B +} \widehat C_B 
	\end{array}\right)
}

% Step 3: Loop-guard 

\renewcommand{\guard}{
	m( A_{TL} ) < m( A )
}

% Step 4: Initialize 

\renewcommand{\partitionings}{
	$
	A \rightarrow
	\left(\begin{array}{c I c} 
	A_{TL} & A_{TR} \\ \whline
	A_{BL} & A_{BR} 
	\end{array}\right) 
	$
	,
	$
	B \rightarrow
	\left(\begin{array}{c}
	B_{T} \\ \whline 
	B_{B}
	\end{array}\right)
	$
	,
	$
	C \rightarrow
	\left(\begin{array}{c}
	C_{T} \\ \whline 
	C_{B}
	\end{array}\right)
	$
}

\renewcommand{\partitionsizes}{
	$ A_{TL} $ is $ 0 \times 0 $,
	$ B_T $ has $ 0 $ rows,
	$ C_T $ has $ 0 $ rows
}

% Step 5a: Repartition the operands 

\renewcommand{\blocksize}{b}

\renewcommand{\repartitionings}{
	$  \left(\begin{array}{c I c}
	A_{TL} & A_{TR} \\ \whline 
	A_{BL} & A_{BR}
	\end{array}\right) 
	\rightarrow
	\left(\begin{array}{c I c c}
	A_{00} & A_{01} & A_{02} \\ \whline 
	A_{10} & A_{11} & A_{12} \\  
	A_{20} & A_{21} & A_{22}
	\end{array}\right) 
	$
	,
	$  \left(\begin{array}{c}
	B_T \\ \whline
	B_B 
	\end{array}\right) 
	\rightarrow
	\left(\begin{array}{c}
	B_0 \\ \whline 
	B_1 \\  
	B_2
	\end{array}\right)
	$
	,
	$  \left(\begin{array}{c}
	C_T \\ \whline
	C_B 
	\end{array}\right) 
	\rightarrow
	\left(\begin{array}{c}
	C_0 \\ \whline 
	C_1 \\  
	C_2
	\end{array}\right)
	$
}

\renewcommand{\repartitionsizes}{
	$ A_{11} $ is $ b \times b $,
	$ B_1 $ has $ b $ rows,
	$ C_1 $ has $ b $ rows}

% Step 5b: Move the double lines 

\renewcommand{\moveboundaries}{
	$  \left(\begin{array}{c I c}
	A_{TL} & A_{TR} \\ \whline 
	A_{BL} & A_{BR}
	\end{array}\right) 
	\leftarrow
	\left(\begin{array}{c c I c}
	A_{00} & A_{01} & A_{02} \\  
	A_{10} & A_{11} & A_{12} \\ \whline 
	A_{20} & A_{21} & A_{22}
	\end{array}\right) 
	$
	,
	$  \left(\begin{array}{c}
	B_T \\ \whline
	B_B 
	\end{array}\right) 
	\leftarrow
	\left(\begin{array}{c}
	B_0 \\  
	B_1 \\ \whline 
	B_2
	\end{array}\right) 
	$
	,
	$  \left(\begin{array}{c}
	C_T \\ \whline
	C_B 
	\end{array}\right) 
	\leftarrow
	\left(\begin{array}{c}
	C_0 \\  
	C_1 \\ \whline 
	C_2
	\end{array}\right) 
	$
}

\renewcommand{\beforeupdate}{$
	\left(\begin{array}{c}
	C_0 \\ 
	C_1 \\
	C_2 
	\end{array}\right)  = 
	\left(\begin{array}{r}
	A_{00} B_0 + A_{10}^T B_1 + A_{20}^T B_2 + \widehat C_0 \\ 
	A_{10} B_0 + {\color{white} A_{11} B_1 + A_{21}^T B_2 +} \widehat C \\ 
	A_{20} B_0 + {\color{white} A_{21} B_1 + A_{22} B_2 +} \widehat C_2 
	\end{array}\right)
	$}


% Step 7: State after update
% Note: The below needs editing consistent with loop-invariant!!!

\renewcommand{\afterupdate}{$ 
	\left(\begin{array}{c}
	C_0 \\ 
	C_1 \\
	C_2 
	\end{array}\right)  = 
	\left(\begin{array}{r}
	A_{00} B_0 + A_{10}^T B_1 + A_{20}^T B_2 + \widehat C_0 \\ 
	A_{10} B_0 + A_{11} B_1 + A_{21}^T B_2 + \widehat C \\ 
	A_{20} B_0 + A_{21} B_1 + {\color{white} A_{22} B_2 +} \widehat C_2 
	\end{array}\right)
	$}


% Step 8: Insert the updates required to change the 
%         state from that given in Step 6 to that given in Step 7
% Note: The below needs editing!!!

\renewcommand{\update}{
	$
	\begin{array}{l}          % do not delete this line 
	{\color{white} C_0 := A_{00} B_0 + A_{10}^T B_1 + A_{20}^T B_2 + C_0} \\ 
	C_1 := {\color{white} A_{10} B_0 +} A_{11} B_1 + A_{21}^T B_1 + C_1 \\ 
	C_2 := {\color{white} A_{20} B_0 +} A_{21} B_1 + {\color{white} A_{22} B_2 +} C_2 \\ 
		\end{array}               % do not delete this line 
	$
}


\begin{center}
	\FlaWorksheet
\end{center}

\newpage

\begin{figure}[p]
\begin{center}
	\FlaWorksheetZero
\end{center}
\end{figure}

\begin{figure}[p]
\begin{center}
	\FlaWorksheetOne
\end{center}
\end{figure}

\begin{figure}[p]
\begin{center}
	\FlaWorksheetTwo
\end{center}
\end{figure}

\begin{figure}[p]
\begin{center}
	\FlaWorksheetThree
\end{center}
\end{figure}

\begin{figure}[p]
	\begin{center}
	\FlaWorksheetFour
\end{center}
\end{figure}

\begin{figure}[p]
	\begin{center}
	\FlaWorksheetFive
\end{center}
\end{figure}

\begin{figure}[p]
	\begin{center}
	\FlaWorksheetSix
\end{center}
\end{figure}

\begin{figure}[p]
	\begin{center}
	\FlaWorksheetSeven
\end{center}
\end{figure}

\begin{figure}[p]
	\begin{center}
	\FlaWorksheetEight
\end{center}
\end{figure}

\begin{figure}[p]
	\begin{center}
	\FlaWorksheetNine
\end{center}
\end{figure}

\begin{figure}[p]
\begin{center}
	\FlaAlgorithm
\end{center}
\end{figure}

\end{document}