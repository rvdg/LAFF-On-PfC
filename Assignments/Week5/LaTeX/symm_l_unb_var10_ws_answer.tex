\documentclass[12pt]{article}

\usepackage{amssymb}
\usepackage{ifthen}
\usepackage[table]{xcolor}
\usepackage{minitoc}
\usepackage{array}

\definecolor{yellow}{cmyk}{0,0,1,0}
\renewcommand{\arraystretch}{1.4}
\newcommand{\R}{\mathbb{R}}

\usepackage{colortbl}

% Page size
\setlength{\oddsidemargin}{-0.5in}
\setlength{\evensidemargin}{-0.5in}
\setlength{\textheight}{10.25in}
\setlength{\textwidth}{7.0in}
\setlength{\topmargin}{-1.35in}

\renewcommand{\arraycolsep}{3pt}


\input color_flatex

\begin{document}
\pagestyle{empty}


\resetsteps % reset all definitions

% Insert output from Spark webpage below


\resetsteps      % Reset all the commands to create a blank worksheet  

% Define the operation to be computed

\renewcommand{\operation}{ C := A B + C \mbox{~~~where $ A $ is symmetric and stored in the lower triangular part}}

\renewcommand{\routinename}{\operation}

% Step 1a: Precondition 

\renewcommand{\precondition}{
	C = \widehat{C}
}

% Step 1b: Postcondition 

\renewcommand{\postcondition}{ 
	 C := A B + \widehat{C} 
}

% Step 2: Invariant 
% Note: Right-hand side of equalities must be updated appropriately

\renewcommand{\invariant}{
	\left(\begin{array}{c I c}
		C_L & C_R
	\end{array}\right)  = 
	\left(\begin{array}{c I c}
		{\color{white} A B_L +} \widehat C_L   & A B_R + \widehat C_R
	\end{array}\right)
}

% Step 3: Loop-guard 

\renewcommand{\guard}{
	n( B_L ) < n( B )
}

% Step 4: Initialize 

\renewcommand{\partitionings}{
	$
	B \rightarrow
	\left(\begin{array}{c I c}
	B_L & B_R
	\end{array}\right)
	$
	,
	$
	C \rightarrow
	\left(\begin{array}{c I c}
	C_L & C_R
	\end{array}\right)
	$
}

\renewcommand{\partitionsizes}{
	$ B_L $ has $ 0 $ columns,
	$ C_L $ has $ 0 $ columns
}

% Step 5a: Repartition the operands 

\renewcommand{\repartitionings}{
	$  \left(\begin{array}{c I c}
	B_L & B_R
	\end{array}\right)
	\rightarrow
	\left(\begin{array}{c I c c}
	B_0 & b_1 & B_2
	\end{array}\right)
	$
	,
	$  \left(\begin{array}{c I c}
	C_L & C_R
	\end{array}\right)
	\rightarrow
	\left(\begin{array}{c I c c}
	C_0 & c_1 & C_2
	\end{array}\right)
	$
}

\renewcommand{\repartitionsizes}{
	$ b_1 $ has $ 1 $ column,
	$ c_1 $ has $ 1 $ column}

% Step 5b: Move the double lines 

\renewcommand{\moveboundaries}{
	$  
	B \rightarrow
	\left(\begin{array}{c I c}
	B_L & B_R
	\end{array}\right)
	\leftarrow
	\left(\begin{array}{c c I c}
	B_0 & b_1 & B_2
	\end{array}\right)
	$
	,
	$  
	C \rightarrow
	\left(\begin{array}{c I c}
	C_L & C_R
	\end{array}\right)
	\leftarrow
	\left(\begin{array}{c c I c}
	C_0 & c_1 & C_2
	\end{array}\right)
	$
}

% Step 6: State before update
% Note: The below needs editing consistent with loop-invariant!!!

\renewcommand{\beforeupdate}{$
	\left(\begin{array}{c c c}
	C_0 & c_1 & C_2
	\end{array}\right)  = 
	\left(\begin{array}{c c c}
	{\color{white} A B_0 +} \widehat C_0   & {\color{white} A b_1 +} \widehat c_1 & A B_2 + \widehat C_2
	\end{array}\right)
	 $}


% Step 7: State after update
% Note: The below needs editing consistent with loop-invariant!!!

\renewcommand{\afterupdate}{$ 
		\left(\begin{array}{c c c}
		C_0 & c_1 & C_2
		\end{array}\right)  = 
		\left(\begin{array}{c c c}
		{\color{white} A B_0 +} \widehat C_0   & A b_1 + \widehat c_1 & A B_2 + \widehat C_2
		\end{array}\right)
	$}


% Step 8: Insert the updates required to change the 
%         state from that given in Step 6 to that given in Step 7
% Note: The below needs editing!!!

\renewcommand{\update}{
	$
	\begin{array}{l}          % do not delete this line 
	c_1 := A b_1 + c_1 
	\end{array}               % do not delete this line 
	$
}



\begin{center}
	\FlaWorksheet
\end{center}

\newpage

\begin{figure}[p]
\begin{center}
	\FlaWorksheetZero
\end{center}
\end{figure}

\begin{figure}[p]
\begin{center}
	\FlaWorksheetOne
\end{center}
\end{figure}

\begin{figure}[p]
\begin{center}
	\FlaWorksheetTwo
\end{center}
\end{figure}

\begin{figure}[p]
\begin{center}
	\FlaWorksheetThree
\end{center}
\end{figure}

\begin{figure}[p]
	\begin{center}
	\FlaWorksheetFour
\end{center}
\end{figure}

\begin{figure}[p]
	\begin{center}
	\FlaWorksheetFive
\end{center}
\end{figure}

\begin{figure}[p]
	\begin{center}
	\FlaWorksheetSix
\end{center}
\end{figure}

\begin{figure}[p]
	\begin{center}
	\FlaWorksheetSeven
\end{center}
\end{figure}

\begin{figure}[p]
	\begin{center}
	\FlaWorksheetEight
\end{center}
\end{figure}

\begin{figure}[p]
	\begin{center}
	\FlaWorksheetNine
\end{center}
\end{figure}

\begin{figure}[p]
\begin{center}
	\FlaAlgorithm
\end{center}
\end{figure}

\end{document}